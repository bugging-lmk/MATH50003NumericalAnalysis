\documentclass[12pt,a4paper]{article}

\usepackage[a4paper,text={16.5cm,25.2cm},centering]{geometry}
\usepackage{lmodern}
\usepackage{amssymb,amsmath}
\usepackage{bm}
\usepackage{graphicx}
\usepackage{microtype}
\usepackage{hyperref}
\usepackage[usenames,dvipsnames]{xcolor}
\setlength{\parindent}{0pt}
\setlength{\parskip}{1.2ex}




\hypersetup
       {   pdfauthor = {  },
           pdftitle={  },
           colorlinks=TRUE,
           linkcolor=black,
           citecolor=blue,
           urlcolor=blue
       }




\usepackage{upquote}
\usepackage{listings}
\usepackage{xcolor}
\lstset{
    basicstyle=\ttfamily\footnotesize,
    upquote=true,
    breaklines=true,
    breakindent=0pt,
    keepspaces=true,
    showspaces=false,
    columns=fullflexible,
    showtabs=false,
    showstringspaces=false,
    escapeinside={(*@}{@*)},
    extendedchars=true,
}
\newcommand{\HLJLt}[1]{#1}
\newcommand{\HLJLw}[1]{#1}
\newcommand{\HLJLe}[1]{#1}
\newcommand{\HLJLeB}[1]{#1}
\newcommand{\HLJLo}[1]{#1}
\newcommand{\HLJLk}[1]{\textcolor[RGB]{148,91,176}{\textbf{#1}}}
\newcommand{\HLJLkc}[1]{\textcolor[RGB]{59,151,46}{\textit{#1}}}
\newcommand{\HLJLkd}[1]{\textcolor[RGB]{214,102,97}{\textit{#1}}}
\newcommand{\HLJLkn}[1]{\textcolor[RGB]{148,91,176}{\textbf{#1}}}
\newcommand{\HLJLkp}[1]{\textcolor[RGB]{148,91,176}{\textbf{#1}}}
\newcommand{\HLJLkr}[1]{\textcolor[RGB]{148,91,176}{\textbf{#1}}}
\newcommand{\HLJLkt}[1]{\textcolor[RGB]{148,91,176}{\textbf{#1}}}
\newcommand{\HLJLn}[1]{#1}
\newcommand{\HLJLna}[1]{#1}
\newcommand{\HLJLnb}[1]{#1}
\newcommand{\HLJLnbp}[1]{#1}
\newcommand{\HLJLnc}[1]{#1}
\newcommand{\HLJLncB}[1]{#1}
\newcommand{\HLJLnd}[1]{\textcolor[RGB]{214,102,97}{#1}}
\newcommand{\HLJLne}[1]{#1}
\newcommand{\HLJLneB}[1]{#1}
\newcommand{\HLJLnf}[1]{\textcolor[RGB]{66,102,213}{#1}}
\newcommand{\HLJLnfm}[1]{\textcolor[RGB]{66,102,213}{#1}}
\newcommand{\HLJLnp}[1]{#1}
\newcommand{\HLJLnl}[1]{#1}
\newcommand{\HLJLnn}[1]{#1}
\newcommand{\HLJLno}[1]{#1}
\newcommand{\HLJLnt}[1]{#1}
\newcommand{\HLJLnv}[1]{#1}
\newcommand{\HLJLnvc}[1]{#1}
\newcommand{\HLJLnvg}[1]{#1}
\newcommand{\HLJLnvi}[1]{#1}
\newcommand{\HLJLnvm}[1]{#1}
\newcommand{\HLJLl}[1]{#1}
\newcommand{\HLJLld}[1]{\textcolor[RGB]{148,91,176}{\textit{#1}}}
\newcommand{\HLJLs}[1]{\textcolor[RGB]{201,61,57}{#1}}
\newcommand{\HLJLsa}[1]{\textcolor[RGB]{201,61,57}{#1}}
\newcommand{\HLJLsb}[1]{\textcolor[RGB]{201,61,57}{#1}}
\newcommand{\HLJLsc}[1]{\textcolor[RGB]{201,61,57}{#1}}
\newcommand{\HLJLsd}[1]{\textcolor[RGB]{201,61,57}{#1}}
\newcommand{\HLJLsdB}[1]{\textcolor[RGB]{201,61,57}{#1}}
\newcommand{\HLJLsdC}[1]{\textcolor[RGB]{201,61,57}{#1}}
\newcommand{\HLJLse}[1]{\textcolor[RGB]{59,151,46}{#1}}
\newcommand{\HLJLsh}[1]{\textcolor[RGB]{201,61,57}{#1}}
\newcommand{\HLJLsi}[1]{#1}
\newcommand{\HLJLso}[1]{\textcolor[RGB]{201,61,57}{#1}}
\newcommand{\HLJLsr}[1]{\textcolor[RGB]{201,61,57}{#1}}
\newcommand{\HLJLss}[1]{\textcolor[RGB]{201,61,57}{#1}}
\newcommand{\HLJLssB}[1]{\textcolor[RGB]{201,61,57}{#1}}
\newcommand{\HLJLnB}[1]{\textcolor[RGB]{59,151,46}{#1}}
\newcommand{\HLJLnbB}[1]{\textcolor[RGB]{59,151,46}{#1}}
\newcommand{\HLJLnfB}[1]{\textcolor[RGB]{59,151,46}{#1}}
\newcommand{\HLJLnh}[1]{\textcolor[RGB]{59,151,46}{#1}}
\newcommand{\HLJLni}[1]{\textcolor[RGB]{59,151,46}{#1}}
\newcommand{\HLJLnil}[1]{\textcolor[RGB]{59,151,46}{#1}}
\newcommand{\HLJLnoB}[1]{\textcolor[RGB]{59,151,46}{#1}}
\newcommand{\HLJLoB}[1]{\textcolor[RGB]{102,102,102}{\textbf{#1}}}
\newcommand{\HLJLow}[1]{\textcolor[RGB]{102,102,102}{\textbf{#1}}}
\newcommand{\HLJLp}[1]{#1}
\newcommand{\HLJLc}[1]{\textcolor[RGB]{153,153,119}{\textit{#1}}}
\newcommand{\HLJLch}[1]{\textcolor[RGB]{153,153,119}{\textit{#1}}}
\newcommand{\HLJLcm}[1]{\textcolor[RGB]{153,153,119}{\textit{#1}}}
\newcommand{\HLJLcp}[1]{\textcolor[RGB]{153,153,119}{\textit{#1}}}
\newcommand{\HLJLcpB}[1]{\textcolor[RGB]{153,153,119}{\textit{#1}}}
\newcommand{\HLJLcs}[1]{\textcolor[RGB]{153,153,119}{\textit{#1}}}
\newcommand{\HLJLcsB}[1]{\textcolor[RGB]{153,153,119}{\textit{#1}}}
\newcommand{\HLJLg}[1]{#1}
\newcommand{\HLJLgd}[1]{#1}
\newcommand{\HLJLge}[1]{#1}
\newcommand{\HLJLgeB}[1]{#1}
\newcommand{\HLJLgh}[1]{#1}
\newcommand{\HLJLgi}[1]{#1}
\newcommand{\HLJLgo}[1]{#1}
\newcommand{\HLJLgp}[1]{#1}
\newcommand{\HLJLgs}[1]{#1}
\newcommand{\HLJLgsB}[1]{#1}
\newcommand{\HLJLgt}[1]{#1}


\def\endash{–}
\def\bbD{ {\mathbb D} }
\def\bbZ{ {\mathbb Z} }
\def\bbR{ {\mathbb R} }
\def\bbC{ {\mathbb C} }

\def\x{ {\vc x} }
\def\a{ {\vc a} }
\def\b{ {\vc b} }
\def\e{ {\vc e} }
\def\f{ {\vc f} }
\def\u{ {\vc u} }
\def\v{ {\vc v} }
\def\y{ {\vc y} }
\def\z{ {\vc z} }
\def\w{ {\vc w} }

\def\bt{ {\tilde b} }
\def\ct{ {\tilde c} }
\def\Ut{ {\tilde U} }
\def\Qt{ {\tilde Q} }
\def\Rt{ {\tilde R} }
\def\Xt{ {\tilde X} }
\def\acos{ {\rm acos}\, }

\def\red#1{ {\color{red} #1} }
\def\blue#1{ {\color{blue} #1} }
\def\green#1{ {\color{ForestGreen} #1} }
\def\magenta#1{ {\color{magenta} #1} }


\def\addtab#1={#1\;&=}

\def\meeq#1{\def\ccr{\\\addtab}
%\tabskip=\@centering
 \begin{align*}
 \addtab#1
 \end{align*}
  }  
  
  \def\leqaddtab#1\leq{#1\;&\leq}
  \def\mleeq#1{\def\ccr{\\\addtab}
%\tabskip=\@centering
 \begin{align*}
 \leqaddtab#1
 \end{align*}
  }  


\def\vc#1{\mbox{\boldmath$#1$\unboldmath}}

\def\vcsmall#1{\mbox{\boldmath$\scriptstyle #1$\unboldmath}}

\def\vczero{{\mathbf 0}}


%\def\beginlist{\begin{itemize}}
%
%\def\endlist{\end{itemize}}


\def\pr(#1){\left({#1}\right)}
\def\br[#1]{\left[{#1}\right]}
\def\fbr[#1]{\!\left[{#1}\right]}
\def\set#1{\left\{{#1}\right\}}
\def\ip<#1>{\left\langle{#1}\right\rangle}
\def\iip<#1>{\left\langle\!\langle{#1}\right\rangle\!\rangle}

\def\norm#1{\left\| #1 \right\|}

\def\abs#1{\left|{#1}\right|}
\def\fpr(#1){\!\pr({#1})}

\def\Re{{\rm Re}\,}
\def\Im{{\rm Im}\,}

\def\floor#1{\left\lfloor#1\right\rfloor}
\def\ceil#1{\left\lceil#1\right\rceil}


\def\mapengine#1,#2.{\mapfunction{#1}\ifx\void#2\else\mapengine #2.\fi }

\def\map[#1]{\mapengine #1,\void.}

\def\mapenginesep_#1#2,#3.{\mapfunction{#2}\ifx\void#3\else#1\mapengine #3.\fi }

\def\mapsep_#1[#2]{\mapenginesep_{#1}#2,\void.}


\def\vcbr{\br}


\def\bvect[#1,#2]{
{
\def\dots{\cdots}
\def\mapfunction##1{\ | \  ##1}
\begin{pmatrix}
		 \,#1\map[#2]\,
\end{pmatrix}
}
}

\def\vect[#1]{
{\def\dots{\ldots}
	\vcbr[{#1}]
}}

\def\vectt[#1]{
{\def\dots{\ldots}
	\vect[{#1}]^{\top}
}}

\def\Vectt[#1]{
{
\def\mapfunction##1{##1 \cr} 
\def\dots{\vdots}
	\begin{pmatrix}
		\map[#1]
	\end{pmatrix}
}}



\def\thetaB{\mbox{\boldmath$\theta$}}
\def\zetaB{\mbox{\boldmath$\zeta$}}


\def\newterm#1{{\it #1}\index{#1}}


\def\TT{{\mathbb T}}
\def\C{{\mathbb C}}
\def\R{{\mathbb R}}
\def\II{{\mathbb I}}
\def\F{{\mathcal F}}
\def\E{{\rm e}}
\def\I{{\rm i}}
\def\D{{\rm d}}
\def\dx{\D x}
\def\CC{{\cal C}}
\def\DD{{\cal D}}
\def\U{{\mathbb U}}
\def\A{{\cal A}}
\def\K{{\cal K}}
\def\DTU{{\cal D}_{{\rm T} \rightarrow {\rm U}}}
\def\LL{{\cal L}}
\def\B{{\cal B}}
\def\T{{\cal T}}
\def\W{{\cal W}}


\def\tF_#1{{\tt F}_{#1}}
\def\Fm{\tF_m}
\def\Fab{\tF_{\alpha,\beta}}
\def\FC{\T}
\def\FCpmz{\FC^{\pm {\rm z}}}
\def\FCz{\FC^{\rm z}}

\def\tFC_#1{{\tt T}_{#1}}
\def\FCn{\tFC_n}

\def\rmz{{\rm z}}

\def\chapref#1{Chapter~\ref{Chapter:#1}}
\def\secref#1{Section~\ref{Section:#1}}
\def\exref#1{Exercise~\ref{Exercise:#1}}
\def\lmref#1{Lemma~\ref{Lemma:#1}}
\def\propref#1{Proposition~\ref{Proposition:#1}}
\def\warnref#1{Warning~\ref{Warning:#1}}
\def\thref#1{Theorem~\ref{Theorem:#1}}
\def\defref#1{Definition~\ref{Definition:#1}}
\def\probref#1{Problem~\ref{Problem:#1}}
\def\corref#1{Corollary~\ref{Corollary:#1}}

\def\sgn{{\rm sgn}\,}
\def\Ai{{\rm Ai}\,}
\def\Bi{{\rm Bi}\,}
\def\wind{{\rm wind}\,}
\def\erf{{\rm erf}\,}
\def\erfc{{\rm erfc}\,}
\def\qqquad{\qquad\quad}
\def\qqqquad{\qquad\qquad}


\def\spand{\hbox{ and }}
\def\spodd{\hbox{ odd}}
\def\speven{\hbox{ even}}
\def\qand{\quad\hbox{and}\quad}
\def\qqand{\qquad\hbox{and}\qquad}
\def\qfor{\quad\hbox{for}\quad}
\def\qqfor{\qquad\hbox{for}\qquad}
\def\qas{\quad\hbox{as}\quad}
\def\qqas{\qquad\hbox{as}\qquad}
\def\qor{\quad\hbox{or}\quad}
\def\qqor{\qquad\hbox{or}\qquad}
\def\qqwhere{\qquad\hbox{where}\qquad}



%%% Words

\def\naive{na\"\i ve\xspace}
\def\Jmap{Joukowsky map\xspace}
\def\Mobius{M\"obius\xspace}
\def\Holder{H\"older\xspace}
\def\Mathematica{{\sc Mathematica}\xspace}
\def\apriori{apriori\xspace}
\def\WHf{Weiner--Hopf factorization\xspace}
\def\WHfs{Weiner--Hopf factorizations\xspace}

\def\Jup{J_\uparrow^{-1}}
\def\Jdown{J_\downarrow^{-1}}
\def\Jin{J_+^{-1}}
\def\Jout{J_-^{-1}}



\def\bD{\D\!\!\!^-}

\def\Abstract#1\par{\begin{abstract}#1\end{abstract}}
\def\Keywords#1\par{\begin{keywords}{#1}\end{keywords}}
\def\Section#1#2.{\section{#2}\label{Section:#1} }
\def\Appendix#1#2.{\appendix \section{#2}\label{Section:#1} }

\def\Subsectionl#1#2.{\subsection{#2}\label{subsec:#1}}
\def\Subsection#1.{\subsection{#1}}

\def\Subsubsection#1.{\subsubsection{#1}}


\def\Problem#1#2\par{\begin{problem}\label{Problem:#1} #2\end{problem}}
\def\Theorem#1#2\par{\begin{theorem}\label{Theorem:#1} #2\end{theorem}}
\def\Conjecture#1#2\par{\begin{conjecture}\label{Conjecture:#1} #2\end{conjecture}}
\def\Proposition#1#2\par{\begin{proposition}\label{Proposition:#1} #2\end{proposition}}
\def\Definition#1#2\par{\begin{definition}\label{Definition:#1} #2\end{definition}}
\def\Corollary#1#2\par{\begin{corollary}\label{Corollary:#1} #2\end{corollary}}
\def\Lemma#1#2\par{\begin{lemma}\label{Lemma:#1} #2\end{lemma}}
\def\Example#1#2\par{\begin{example}\label{Example:#1} #2\end{example}}
\def\Remark #1\par{\begin{remark*}#1\end{remark*}}

\def\figref#1{Figure~\ref{fig:#1}}

\def\Figurew[#1]#2#3\par{
\begin{figure}[tb]
\begin{center}{
\includegraphics[width=#2]{Figures/#1}}
\end{center}
\caption{#3}\label{fig:#1} 
\end{figure}
}

\def\Figure[#1]#2\par{
\begin{figure}[tb]
\begin{center}{
\includegraphics{Figures/#1}}
\end{center}
\caption{#2}\label{fig:#1} 
\end{figure}
}

\def\Figurefixed[#1]#2\par{
\Figurew[#1]{0.48 \hsize}{#2}\par
}

\def\Figuretwow#1#2#3#4\par{
\begin{figure}[tb]
\begin{center}{
\includegraphics[width=#3]{Figures/#1}\includegraphics[width=#3]{Figures/#2}}
\end{center}
\caption{#4}\label{fig:#1} 
\end{figure}
}

\def\Figuretwowframed#1#2#3#4\par{
\begin{figure}[tb]
\begin{center}{
\fbox{\includegraphics[width=#3]{Figures/#1}}\fbox{\includegraphics[width=#3]{Figures/#2}}}
\end{center}
\caption{#4}\label{fig:#1} 
\end{figure}
}

\def\Figuretwo[#1,#2]#3\par{
	\Figuretwow{#1}{#2}{0.48 \hsize}
		#3\par	
}

\def\Figuretwoframed[#1,#2]#3\par{
	\Figuretwowframed{#1}{#2}{0.48 \hsize}
		#3\par	
}

\def\Figurethreew#1#2#3#4#5\par{
\begin{figure}[tb]
\begin{center}{
\includegraphics[width=#4]{Figures/#1} \includegraphics[width=#4]{Figures/#2} \includegraphics[width=#4]{Figures/#3}}
\end{center}
\caption{#5}\label{fig:#1} %\prooflabel{#1}
\end{figure}
}

\def\Figurethree#1#2#3#4\par{
	\Figurethreew{#1}{#2}{#3}{0.3 \hsize}
		{#4}\par	
}

\def\Figurematrixfour#1#2#3#4#5\par{
\begin{figure}[tb]
\begin{center}{
\vbox{\hbox{\includegraphics[width= 0.48 \hsize]{Figures/#1} \includegraphics[width= 0.48 \hsize]{Figures/#2}}\hbox{\includegraphics[width= 0.48 \hsize]{Figures/#3}\includegraphics[width= 0.48 \hsize]{Figures/#4}}}}
\end{center}
\caption{#5}\label{fig:#1} %\prooflabel{#1}
\end{figure}
}


\def\questionequals{= \!\!\!\!\!\!{\scriptstyle ? \atop }\,\,\,}

\def\elll#1{\ell^{\lambda,#1}}
\def\elllp{\ell^{\lambda,p}}
\def\elllRp{\ell^{(\lambda,R),p}}


\def\elllRpz_#1{\ell_{#1{\rm z}}^{(\lambda,R),p}}


\def\sopmatrix#1{\begin{pmatrix}#1\end{pmatrix}}

\def\Proof{\begin{proof}}
\def\mqed{\end{proof}}

\gdef\reffilename{\jobname}
\def\References{\bibliography{\reffilename}}

\outer\def\ends{ 
\end{document}
}


\begin{document}



\textbf{Numerical Analysis MATH50003 (2024\ensuremath{\endash}25) Problem Sheet 7}

\textbf{Problem 1} Use Lagrange interpolation to interpolate the function $\cos x$ by a polynomial at the points $[0,2,3,4]$ and evaluate at $x = 1$.

\textbf{SOLUTION}

\begin{itemize}
\item \[
\ensuremath{\ell}_1(x)=\frac{(x-2)(x-3)(x-4)}{(0-2)(0-3)(0-4)}=-\frac{1}{24}(x-2)(x-3)(x-4)
\]

\item \[
\ensuremath{\ell}_2(x)=\frac{(x-0)(x-3)(x-4)}{(2-0)(2-3)(2-4)}=\frac{1}{4}x(x-3)(x-4)
\]

\item \[
\ensuremath{\ell}_3(x)=\frac{(x-0)(x-2)(x-4)}{(3-0)(3-2)(3-4)}=-\frac{1}{3}x(x-2)(x-4)
\]

\item \[
\ensuremath{\ell}_4(x)=\frac{(x-0)(x-2)(x-3)}{(4-0)(4-2)(4-3)}=\frac{1}{8}x(x-2)(x-3)
\]
\end{itemize}
So that $p(x)=\cos(0)\ensuremath{\ell}_1(x)+\cos(2)\ensuremath{\ell}_2(x)+\cos(3)\ensuremath{\ell}_3(x)+\cos(4)\ensuremath{\ell}_4(x)$. Note that $\ensuremath{\ell}_0(1)=1/4$, $\ensuremath{\ell}_2(1)=3/2$, $\ensuremath{\ell}_3(1)=-1$, $\ensuremath{\ell}_4(1)=1/4$, so $p(1)=1/4\cos(0)+3/2\cos(2)-\cos(3)+1/4\cos(4)$.

\textbf{END}

\textbf{Problem 2} Compute the LU factorisation of the following transposed Vandermonde matrices:
\[
\begin{bmatrix}
1 & 1 \\
x & y
\end{bmatrix},
\begin{bmatrix}
1 & 1 & 1 \\
x & y & z \\
x^2 & y^2 & z^2
\end{bmatrix},
\begin{bmatrix}
1 & 1 & 1 & 1 \\
x & y & z & t \\
x^2 & y^2 & z^2 & t^2 \\
x^3 & y^3 & z^3 & t^3
\end{bmatrix}
\]
Can you spot a pattern? Test your conjecture with a $5 \ensuremath{\times} 5$ Vandermonde matrix.

\textbf{SOLUTION} (1)
\[
\begin{bmatrix}
1 & 1 \\
x & y
\end{bmatrix} =  \begin{bmatrix}
1 &  \\
x & 1
\end{bmatrix} \begin{bmatrix}
1 & 1 \\
 & y-x
\end{bmatrix}
\]
(2)
\[
V := \begin{bmatrix}
1 & 1 & 1 \\
x & y & z \\
x^2 & y^2 & z^2
\end{bmatrix} =  \begin{bmatrix}
1 &  \\
x & 1 \\
x^2 && 1
\end{bmatrix} \begin{bmatrix}
1 & 1 & 1\\
 & y-x & z-x \\
 & y^2-x^2 & z^2 - x^2
\end{bmatrix}
\]
We then have
\[
\begin{bmatrix}
 y-x & z-x \\
 y^2-x^2 & z^2 - x^2
\end{bmatrix} = \begin{bmatrix}
 1 &  \\
 y+x & 1
\end{bmatrix} \begin{bmatrix}
y-x & z-x \\
& (z-y)(z-x)
\end{bmatrix}
\]
since $z^2 - x^2 - (z-x) (y+x) = z^2 + xy  - zy = (z-y)(z-x)$. Thus we have
\[
V = \begin{bmatrix}
1 &  \\
x & 1 \\
x^2 & x+y& 1
\end{bmatrix}  \begin{bmatrix}
1 & 1 & 1\\
 & y-x & z-x \\
 &  &  (z-y)(z-x)
\end{bmatrix}
\]
(3)
\[
V := \begin{bmatrix}
1 & 1 & 1 & 1 \\
x & y & z & t \\
x^2 & y^2 & z^2 & t^2 \\
x^3 & y^3 & z^3 & t^3
\end{bmatrix} =
\begin{bmatrix}
1 &  \\
x & 1 \\
x^2 && 1 \\
x^3 &&& 1
\end{bmatrix} \begin{bmatrix}
1 & 1 & 1 & 1\\
 & y-x & z-x & t-x \\
 & y^2-x^2 & z^2 - x^2 & t^2 - x^2 \\
 & y^3-x^3 & z^3 - x^3 & t^3 - x^3
\end{bmatrix}
\]
We then have
\meeq{
\begin{bmatrix}
y-x & z-x & t-x \\
y^2-x^2 & z^2 - x^2 & t^2 - x^2 \\
y^3-x^3 & z^3 - x^3 & t^3 - x^3
\end{bmatrix} = \begin{bmatrix}
1 &  &  \\
y + x & 1 &  \\
y^2 + xy + x^2 &  & 1
\end{bmatrix} \\
& \qquad \ensuremath{\times} \begin{bmatrix}
y-x & z-x & t-x \\
 & (z-y)(z-x) & (t-y)(t-x) \\
 & (z-x)(z-y) (x+y+z) & (t-x)(t-y) (x+y+t)
\end{bmatrix}
}
since
\[
z^3 - x^3 - (z-x) (y^2 + x y + x^2) = z^3 - z y^2 - x y z  - z x^2 + x y^2  + x^2 y
= (x-z)(y-z) (x+y+z).
\]
Finally we have
\begin{align*}
&\begin{bmatrix}
 (z-y)(z-x) & (t-y)(t-x) \\
 (z-x)(z-y) (x+y+z) & (t-x)(t-y) (x+y+t)
\end{bmatrix}\\
&\qquad = \begin{bmatrix}
 1 & \\
 x+y+z & 1
\end{bmatrix}
 \begin{bmatrix}
 (z-y)(z-x) & (t-y)(t-x) \\
  & (t-x)(t-y) (t-z)
\end{bmatrix}
\end{align*}
since
\[
(t-x)(t-y) (x+y+t) - (x+y+z) (t-y)(t-x) = (t-y)(t-x)(t-z).
\]
Putting everything together we have
\[
V = \begin{bmatrix}
1 &  \\
x & 1 \\
x^2 & x+y  & 1 \\
x^3 &y^2 + xy + x^2  & x+ y + z & 1
\end{bmatrix} \begin{bmatrix}
1 & 1 & 1 & 1\\
 & y-x & z-x & t-x \\
 &  & (z-y)(z-x) & (t-y)(t-x) \\
 &  &  & (t-y)(t-x)(t-z)
\end{bmatrix}
\]
We conjecture that $L[k,j]$ for $k > j$ contains a sum of all monomials of degree $k$ of $x_1,\ensuremath{\ldots},x_j$, and
\[
U[k,j] = \ensuremath{\prod}_{s = 1}^{k-1} (x_j-x_s)
\]
for $1 < k \ensuremath{\leq} j$. We can confirm that
\begin{align*}
&\begin{bmatrix}
1 & 1 & 1 & 1 & 1 \\
x & y & z & t & s\\
x^2 & y^2 & z^2 & t^2 & s^2 \\
x^3 & y^3 & z^3 & t^3 & s^3 \\
x^4 & y^4 & z^4 & t^4 & s^4
\end{bmatrix} \\
&\qquad =
\begin{bmatrix}
1 &  \\
x & 1 \\
x^2 & x+y  & 1 \\
x^3 &y^2 + xy + x^2  & x+ y + z & 1 \\
x^4 & x^3 + x^2 y + x y^2 + y^3 & x^2 + y^2 + z^2 + xy + xz + yz  & x + y + z + t & 1
\end{bmatrix} \\
&\qquad \ensuremath{\times}
\begin{bmatrix}
1 & 1 & 1 & 1 & 1\\
 & y-x & z-x & t-x & s-x  \\
 &  & (z-y)(z-x) & (t-y)(t-x) & (s-x) (s-y) \\
 &  &  & (t-y)(t-x)(t-z) &   (s-y)(s-x)(s-z)  \\
 &  &  &  &   (s-y)(s-x)(s-z)(s-t)
\end{bmatrix}
\end{align*}
Multiplying it out we confirm that our conjecture is correct in this case.

\textbf{END}

\textbf{Problem 3} Compute the interpolatory quadrature rule
\[
\ensuremath{\int}_{-1}^1 f(x) w(x) \dx \ensuremath{\approx} \ensuremath{\sum}_{j=1}^n w_j f(x_j)
\]
for the points $[x_1,x_2,x_3] = [-1,1/2,1]$, for the weights $w(x) = 1$ and $w(x) = \sqrt{1-x^2}$.

\textbf{SOLUTION}

\ensuremath{\bullet} $w(x) = 1$

\ensuremath{\bullet} $w(x) = \sqrt{1-x^2}$

For the points $\ensuremath{\bm{\x}} = \{-1, 1/2, 1\}$ we have the Lagrange polynomials:
\[
\ensuremath{\ell}_1(x) = \left(\frac{x - 1/2}{-1 - 1/2}\right)\cdot\left(\frac{x - 1}{-1 - 1}\right) = \frac{1}{3}\left(x^2 - \frac{3}{2}x + \frac{1}{2}\right),
\]
and
\[
\ensuremath{\ell}_2(x) = -\frac{4}{3}x^2 + \frac{4}{3}, \ensuremath{\ell}_3(x) =x^2 + \frac{1}{2}x - \frac{1}{2},
\]
similarly. We can then compute the weights,
\[
w_j = \int_{-1}^1 \ensuremath{\ell}_j(x)w(x)dx,
\]
using,
\[
\int_{-1}^1 x^k \sqrt{1-x^2}dx = \begin{cases}
 \frac{\ensuremath{\pi}}{2} &	k=0 \\
 0 & k=1 \\
\frac{\ensuremath{\pi}}{8} & k=2
 \end{cases}
\]
to find,
\[
w_j = \begin{cases}
 	\frac{\ensuremath{\pi}}{8} & j = 1 \\
 	\frac{\ensuremath{\pi}}{2} & j = 2 \\
 	-\frac{\ensuremath{\pi}}{8} & j = 3,
 \end{cases}
\]
so that the interpolatory quadrature rule is:
\[
\ensuremath{\Sigma}_3^{w,\ensuremath{\bm{\x}}}(f) = \frac{\ensuremath{\pi}}{2}\left(\frac{1}{4}f(-1) + f(1/2) -\frac{1}{4}f(1) \right)
\]
\textbf{END}

\rule{\textwidth}{1pt}
\textbf{Problem 4} Derive  Backward Euler: use the left-sided divided difference approximation
\[
u'(x) \ensuremath{\approx} {u(x) - u(x-h)  \over h}
\]
to reduce the first order ODE
\meeq{
u(a) =  c, \qquad u'(x) + \ensuremath{\omega}(x) u(x) = f(x)
}
to a lower triangular system by discretising on the grid $x_j = a + j h$ for $h = (b-a)/n$. Hint: only impose the ODE on the gridpoints $x_1,\ensuremath{\ldots},x_n$ so that the divided difference does not depend on behaviour at $x_{-1}$.

\textbf{SOLUTION}

We go through all 4 steps (this is to help you understand what to do. In an exam I will still give full credit if you get the right result, even if you don't write down all 4 steps):

(Step 1) Since we need to avoid going off the left in step 2 we start the ODE discretisation at $x_1$:
\[
\Vectt[u(x_0), u'(x_1) +  \ensuremath{\omega}(x_1)u(x_1), \ensuremath{\vdots}, u'(x_n) +  \ensuremath{\omega}(x_n)u(x_n)] = \underbrace{\Vectt[c, f(x_1),\ensuremath{\vdots},f(x_n)]}_{\ensuremath{\bm{\b}}}
\]
(Step 2) Replace with divided differences:
\[
\Vectt[u(x_0), (u(x_1)-u(x_0))/h +  \ensuremath{\omega}(x_1)u(x_1), \ensuremath{\vdots}, (u(x_n)-u(x_{n-1}))/h +  \ensuremath{\omega}(x_n)u(x_n)] \ensuremath{\approx} \ensuremath{\bm{\b}}
\]
(Step 3) Replace with discrete system with equality:
\[
\Vectt[u_0, (u_1-u_0)/h +  \ensuremath{\omega}(x_1)u_1, \ensuremath{\vdots}, (u_n-u_{n-1}))/h +  \ensuremath{\omega}(x_n)u_n] = \ensuremath{\bm{\b}}
\]
(Step 4) Write as linear system:
\[
\begin{bmatrix}
1 \\
-1/h & 1/h + \ensuremath{\omega}(x_1) \\
& \ensuremath{\ddots} & \ensuremath{\ddots} \\
&& -1/h & 1/h + \ensuremath{\omega}(x_n)
\end{bmatrix} \Vectt[u_0,\ensuremath{\vdots},u_n] = \ensuremath{\bm{\b}}
\]
\textbf{END}

\textbf{Problem 5} Reduce a Schrödinger equation to a tridiagonal linear system by discretising on the grid $x_j = a + j h$ for $h = (b-a)/n$:
\meeq{
u(a) =  c,\qquad u''(x) + V(x) u(x) = f(x), \qquad u(b) = d.
}
\textbf{SOLUTION}

(Step 1) 
\[
\Vectt[u(x_0), u''(x_1) + V(x_1) u(x_1), \ensuremath{\vdots}, u'(x_{n-1}) + V(x_{n-1}) u(x_{n-1}), u(x_n)] = \underbrace{\Vectt[c, f(x_1),\ensuremath{\vdots},f(x_{n-1}), d]}_{\ensuremath{\bm{\b}}}
\]
(Step 2) Replace with divided differences:
\[
\Vectt[u(x_0), (u(x_0)-2u(x_1)+u(x_2))/h^2 + V(x_1)u(x_1), \ensuremath{\vdots}, (u(x_{n-2} - 2u(x_{n-1})+u(x_n))/h^2 + V(x_{n-1})u(x_{n-1}), u(x_n)] \ensuremath{\approx} \ensuremath{\bm{\b}}
\]
(Step 3) Replace with discrete system with equality:
\[
\Vectt[u_0, (u_0-2u_1+u_2)/h^2 + V(x_1)u_1, \ensuremath{\vdots}, (u_{n-2}-2u_{n-1}+u_n))/h^2+ V(x_{n-1})u_{n-1},u_n] = \ensuremath{\bm{\b}}
\]
(Step 4) Write as a tridiagonal linear system:
\[
\begin{bmatrix}
1 \\
1/h^2 & V(x_1)-2/h^2 & 1/h^2 \\
& \ensuremath{\ddots} & \ensuremath{\ddots} & \ensuremath{\ddots} \\
&&1/h^2 & V(x_{n-1})-2/h^2 & 1/h^2 \\
&& &&1
\end{bmatrix} \Vectt[u_0,\ensuremath{\vdots},u_n] = \ensuremath{\bm{\b}}
\]
\textbf{END}



\end{document}