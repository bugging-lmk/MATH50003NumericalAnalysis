\documentclass[12pt,a4paper]{article}

\usepackage[a4paper,text={16.5cm,25.2cm},centering]{geometry}
\usepackage{lmodern}
\usepackage{amssymb,amsmath}
\usepackage{bm}
\usepackage{graphicx}
\usepackage{microtype}
\usepackage{hyperref}
\usepackage[usenames,dvipsnames]{xcolor}
\setlength{\parindent}{0pt}
\setlength{\parskip}{1.2ex}




\hypersetup
       {   pdfauthor = {  },
           pdftitle={  },
           colorlinks=TRUE,
           linkcolor=black,
           citecolor=blue,
           urlcolor=blue
       }




\usepackage{upquote}
\usepackage{listings}
\usepackage{xcolor}
\lstset{
    basicstyle=\ttfamily\footnotesize,
    upquote=true,
    breaklines=true,
    breakindent=0pt,
    keepspaces=true,
    showspaces=false,
    columns=fullflexible,
    showtabs=false,
    showstringspaces=false,
    escapeinside={(*@}{@*)},
    extendedchars=true,
}
\newcommand{\HLJLt}[1]{#1}
\newcommand{\HLJLw}[1]{#1}
\newcommand{\HLJLe}[1]{#1}
\newcommand{\HLJLeB}[1]{#1}
\newcommand{\HLJLo}[1]{#1}
\newcommand{\HLJLk}[1]{\textcolor[RGB]{148,91,176}{\textbf{#1}}}
\newcommand{\HLJLkc}[1]{\textcolor[RGB]{59,151,46}{\textit{#1}}}
\newcommand{\HLJLkd}[1]{\textcolor[RGB]{214,102,97}{\textit{#1}}}
\newcommand{\HLJLkn}[1]{\textcolor[RGB]{148,91,176}{\textbf{#1}}}
\newcommand{\HLJLkp}[1]{\textcolor[RGB]{148,91,176}{\textbf{#1}}}
\newcommand{\HLJLkr}[1]{\textcolor[RGB]{148,91,176}{\textbf{#1}}}
\newcommand{\HLJLkt}[1]{\textcolor[RGB]{148,91,176}{\textbf{#1}}}
\newcommand{\HLJLn}[1]{#1}
\newcommand{\HLJLna}[1]{#1}
\newcommand{\HLJLnb}[1]{#1}
\newcommand{\HLJLnbp}[1]{#1}
\newcommand{\HLJLnc}[1]{#1}
\newcommand{\HLJLncB}[1]{#1}
\newcommand{\HLJLnd}[1]{\textcolor[RGB]{214,102,97}{#1}}
\newcommand{\HLJLne}[1]{#1}
\newcommand{\HLJLneB}[1]{#1}
\newcommand{\HLJLnf}[1]{\textcolor[RGB]{66,102,213}{#1}}
\newcommand{\HLJLnfm}[1]{\textcolor[RGB]{66,102,213}{#1}}
\newcommand{\HLJLnp}[1]{#1}
\newcommand{\HLJLnl}[1]{#1}
\newcommand{\HLJLnn}[1]{#1}
\newcommand{\HLJLno}[1]{#1}
\newcommand{\HLJLnt}[1]{#1}
\newcommand{\HLJLnv}[1]{#1}
\newcommand{\HLJLnvc}[1]{#1}
\newcommand{\HLJLnvg}[1]{#1}
\newcommand{\HLJLnvi}[1]{#1}
\newcommand{\HLJLnvm}[1]{#1}
\newcommand{\HLJLl}[1]{#1}
\newcommand{\HLJLld}[1]{\textcolor[RGB]{148,91,176}{\textit{#1}}}
\newcommand{\HLJLs}[1]{\textcolor[RGB]{201,61,57}{#1}}
\newcommand{\HLJLsa}[1]{\textcolor[RGB]{201,61,57}{#1}}
\newcommand{\HLJLsb}[1]{\textcolor[RGB]{201,61,57}{#1}}
\newcommand{\HLJLsc}[1]{\textcolor[RGB]{201,61,57}{#1}}
\newcommand{\HLJLsd}[1]{\textcolor[RGB]{201,61,57}{#1}}
\newcommand{\HLJLsdB}[1]{\textcolor[RGB]{201,61,57}{#1}}
\newcommand{\HLJLsdC}[1]{\textcolor[RGB]{201,61,57}{#1}}
\newcommand{\HLJLse}[1]{\textcolor[RGB]{59,151,46}{#1}}
\newcommand{\HLJLsh}[1]{\textcolor[RGB]{201,61,57}{#1}}
\newcommand{\HLJLsi}[1]{#1}
\newcommand{\HLJLso}[1]{\textcolor[RGB]{201,61,57}{#1}}
\newcommand{\HLJLsr}[1]{\textcolor[RGB]{201,61,57}{#1}}
\newcommand{\HLJLss}[1]{\textcolor[RGB]{201,61,57}{#1}}
\newcommand{\HLJLssB}[1]{\textcolor[RGB]{201,61,57}{#1}}
\newcommand{\HLJLnB}[1]{\textcolor[RGB]{59,151,46}{#1}}
\newcommand{\HLJLnbB}[1]{\textcolor[RGB]{59,151,46}{#1}}
\newcommand{\HLJLnfB}[1]{\textcolor[RGB]{59,151,46}{#1}}
\newcommand{\HLJLnh}[1]{\textcolor[RGB]{59,151,46}{#1}}
\newcommand{\HLJLni}[1]{\textcolor[RGB]{59,151,46}{#1}}
\newcommand{\HLJLnil}[1]{\textcolor[RGB]{59,151,46}{#1}}
\newcommand{\HLJLnoB}[1]{\textcolor[RGB]{59,151,46}{#1}}
\newcommand{\HLJLoB}[1]{\textcolor[RGB]{102,102,102}{\textbf{#1}}}
\newcommand{\HLJLow}[1]{\textcolor[RGB]{102,102,102}{\textbf{#1}}}
\newcommand{\HLJLp}[1]{#1}
\newcommand{\HLJLc}[1]{\textcolor[RGB]{153,153,119}{\textit{#1}}}
\newcommand{\HLJLch}[1]{\textcolor[RGB]{153,153,119}{\textit{#1}}}
\newcommand{\HLJLcm}[1]{\textcolor[RGB]{153,153,119}{\textit{#1}}}
\newcommand{\HLJLcp}[1]{\textcolor[RGB]{153,153,119}{\textit{#1}}}
\newcommand{\HLJLcpB}[1]{\textcolor[RGB]{153,153,119}{\textit{#1}}}
\newcommand{\HLJLcs}[1]{\textcolor[RGB]{153,153,119}{\textit{#1}}}
\newcommand{\HLJLcsB}[1]{\textcolor[RGB]{153,153,119}{\textit{#1}}}
\newcommand{\HLJLg}[1]{#1}
\newcommand{\HLJLgd}[1]{#1}
\newcommand{\HLJLge}[1]{#1}
\newcommand{\HLJLgeB}[1]{#1}
\newcommand{\HLJLgh}[1]{#1}
\newcommand{\HLJLgi}[1]{#1}
\newcommand{\HLJLgo}[1]{#1}
\newcommand{\HLJLgp}[1]{#1}
\newcommand{\HLJLgs}[1]{#1}
\newcommand{\HLJLgsB}[1]{#1}
\newcommand{\HLJLgt}[1]{#1}


\def\endash{–}
\def\bbD{ {\mathbb D} }
\def\bbZ{ {\mathbb Z} }
\def\bbR{ {\mathbb R} }
\def\bbC{ {\mathbb C} }

\def\x{ {\vc x} }
\def\a{ {\vc a} }
\def\b{ {\vc b} }
\def\e{ {\vc e} }
\def\f{ {\vc f} }
\def\u{ {\vc u} }
\def\v{ {\vc v} }
\def\y{ {\vc y} }
\def\z{ {\vc z} }
\def\w{ {\vc w} }

\def\bt{ {\tilde b} }
\def\ct{ {\tilde c} }
\def\Ut{ {\tilde U} }
\def\Qt{ {\tilde Q} }
\def\Rt{ {\tilde R} }
\def\Xt{ {\tilde X} }
\def\acos{ {\rm acos}\, }

\def\red#1{ {\color{red} #1} }
\def\blue#1{ {\color{blue} #1} }
\def\green#1{ {\color{ForestGreen} #1} }
\def\magenta#1{ {\color{magenta} #1} }


\def\addtab#1={#1\;&=}

\def\meeq#1{\def\ccr{\\\addtab}
%\tabskip=\@centering
 \begin{align*}
 \addtab#1
 \end{align*}
  }  
  
  \def\leqaddtab#1\leq{#1\;&\leq}
  \def\mleeq#1{\def\ccr{\\\addtab}
%\tabskip=\@centering
 \begin{align*}
 \leqaddtab#1
 \end{align*}
  }  


\def\vc#1{\mbox{\boldmath$#1$\unboldmath}}

\def\vcsmall#1{\mbox{\boldmath$\scriptstyle #1$\unboldmath}}

\def\vczero{{\mathbf 0}}


%\def\beginlist{\begin{itemize}}
%
%\def\endlist{\end{itemize}}


\def\pr(#1){\left({#1}\right)}
\def\br[#1]{\left[{#1}\right]}
\def\fbr[#1]{\!\left[{#1}\right]}
\def\set#1{\left\{{#1}\right\}}
\def\ip<#1>{\left\langle{#1}\right\rangle}
\def\iip<#1>{\left\langle\!\langle{#1}\right\rangle\!\rangle}

\def\norm#1{\left\| #1 \right\|}

\def\abs#1{\left|{#1}\right|}
\def\fpr(#1){\!\pr({#1})}

\def\Re{{\rm Re}\,}
\def\Im{{\rm Im}\,}

\def\floor#1{\left\lfloor#1\right\rfloor}
\def\ceil#1{\left\lceil#1\right\rceil}


\def\mapengine#1,#2.{\mapfunction{#1}\ifx\void#2\else\mapengine #2.\fi }

\def\map[#1]{\mapengine #1,\void.}

\def\mapenginesep_#1#2,#3.{\mapfunction{#2}\ifx\void#3\else#1\mapengine #3.\fi }

\def\mapsep_#1[#2]{\mapenginesep_{#1}#2,\void.}


\def\vcbr{\br}


\def\bvect[#1,#2]{
{
\def\dots{\cdots}
\def\mapfunction##1{\ | \  ##1}
\begin{pmatrix}
		 \,#1\map[#2]\,
\end{pmatrix}
}
}

\def\vect[#1]{
{\def\dots{\ldots}
	\vcbr[{#1}]
}}

\def\vectt[#1]{
{\def\dots{\ldots}
	\vect[{#1}]^{\top}
}}

\def\Vectt[#1]{
{
\def\mapfunction##1{##1 \cr} 
\def\dots{\vdots}
	\begin{pmatrix}
		\map[#1]
	\end{pmatrix}
}}



\def\thetaB{\mbox{\boldmath$\theta$}}
\def\zetaB{\mbox{\boldmath$\zeta$}}


\def\newterm#1{{\it #1}\index{#1}}


\def\TT{{\mathbb T}}
\def\C{{\mathbb C}}
\def\R{{\mathbb R}}
\def\II{{\mathbb I}}
\def\F{{\mathcal F}}
\def\E{{\rm e}}
\def\I{{\rm i}}
\def\D{{\rm d}}
\def\dx{\D x}
\def\CC{{\cal C}}
\def\DD{{\cal D}}
\def\U{{\mathbb U}}
\def\A{{\cal A}}
\def\K{{\cal K}}
\def\DTU{{\cal D}_{{\rm T} \rightarrow {\rm U}}}
\def\LL{{\cal L}}
\def\B{{\cal B}}
\def\T{{\cal T}}
\def\W{{\cal W}}


\def\tF_#1{{\tt F}_{#1}}
\def\Fm{\tF_m}
\def\Fab{\tF_{\alpha,\beta}}
\def\FC{\T}
\def\FCpmz{\FC^{\pm {\rm z}}}
\def\FCz{\FC^{\rm z}}

\def\tFC_#1{{\tt T}_{#1}}
\def\FCn{\tFC_n}

\def\rmz{{\rm z}}

\def\chapref#1{Chapter~\ref{Chapter:#1}}
\def\secref#1{Section~\ref{Section:#1}}
\def\exref#1{Exercise~\ref{Exercise:#1}}
\def\lmref#1{Lemma~\ref{Lemma:#1}}
\def\propref#1{Proposition~\ref{Proposition:#1}}
\def\warnref#1{Warning~\ref{Warning:#1}}
\def\thref#1{Theorem~\ref{Theorem:#1}}
\def\defref#1{Definition~\ref{Definition:#1}}
\def\probref#1{Problem~\ref{Problem:#1}}
\def\corref#1{Corollary~\ref{Corollary:#1}}

\def\sgn{{\rm sgn}\,}
\def\Ai{{\rm Ai}\,}
\def\Bi{{\rm Bi}\,}
\def\wind{{\rm wind}\,}
\def\erf{{\rm erf}\,}
\def\erfc{{\rm erfc}\,}
\def\qqquad{\qquad\quad}
\def\qqqquad{\qquad\qquad}


\def\spand{\hbox{ and }}
\def\spodd{\hbox{ odd}}
\def\speven{\hbox{ even}}
\def\qand{\quad\hbox{and}\quad}
\def\qqand{\qquad\hbox{and}\qquad}
\def\qfor{\quad\hbox{for}\quad}
\def\qqfor{\qquad\hbox{for}\qquad}
\def\qas{\quad\hbox{as}\quad}
\def\qqas{\qquad\hbox{as}\qquad}
\def\qor{\quad\hbox{or}\quad}
\def\qqor{\qquad\hbox{or}\qquad}
\def\qqwhere{\qquad\hbox{where}\qquad}



%%% Words

\def\naive{na\"\i ve\xspace}
\def\Jmap{Joukowsky map\xspace}
\def\Mobius{M\"obius\xspace}
\def\Holder{H\"older\xspace}
\def\Mathematica{{\sc Mathematica}\xspace}
\def\apriori{apriori\xspace}
\def\WHf{Weiner--Hopf factorization\xspace}
\def\WHfs{Weiner--Hopf factorizations\xspace}

\def\Jup{J_\uparrow^{-1}}
\def\Jdown{J_\downarrow^{-1}}
\def\Jin{J_+^{-1}}
\def\Jout{J_-^{-1}}



\def\bD{\D\!\!\!^-}

\def\Abstract#1\par{\begin{abstract}#1\end{abstract}}
\def\Keywords#1\par{\begin{keywords}{#1}\end{keywords}}
\def\Section#1#2.{\section{#2}\label{Section:#1} }
\def\Appendix#1#2.{\appendix \section{#2}\label{Section:#1} }

\def\Subsectionl#1#2.{\subsection{#2}\label{subsec:#1}}
\def\Subsection#1.{\subsection{#1}}

\def\Subsubsection#1.{\subsubsection{#1}}


\def\Problem#1#2\par{\begin{problem}\label{Problem:#1} #2\end{problem}}
\def\Theorem#1#2\par{\begin{theorem}\label{Theorem:#1} #2\end{theorem}}
\def\Conjecture#1#2\par{\begin{conjecture}\label{Conjecture:#1} #2\end{conjecture}}
\def\Proposition#1#2\par{\begin{proposition}\label{Proposition:#1} #2\end{proposition}}
\def\Definition#1#2\par{\begin{definition}\label{Definition:#1} #2\end{definition}}
\def\Corollary#1#2\par{\begin{corollary}\label{Corollary:#1} #2\end{corollary}}
\def\Lemma#1#2\par{\begin{lemma}\label{Lemma:#1} #2\end{lemma}}
\def\Example#1#2\par{\begin{example}\label{Example:#1} #2\end{example}}
\def\Remark #1\par{\begin{remark*}#1\end{remark*}}

\def\figref#1{Figure~\ref{fig:#1}}

\def\Figurew[#1]#2#3\par{
\begin{figure}[tb]
\begin{center}{
\includegraphics[width=#2]{Figures/#1}}
\end{center}
\caption{#3}\label{fig:#1} 
\end{figure}
}

\def\Figure[#1]#2\par{
\begin{figure}[tb]
\begin{center}{
\includegraphics{Figures/#1}}
\end{center}
\caption{#2}\label{fig:#1} 
\end{figure}
}

\def\Figurefixed[#1]#2\par{
\Figurew[#1]{0.48 \hsize}{#2}\par
}

\def\Figuretwow#1#2#3#4\par{
\begin{figure}[tb]
\begin{center}{
\includegraphics[width=#3]{Figures/#1}\includegraphics[width=#3]{Figures/#2}}
\end{center}
\caption{#4}\label{fig:#1} 
\end{figure}
}

\def\Figuretwowframed#1#2#3#4\par{
\begin{figure}[tb]
\begin{center}{
\fbox{\includegraphics[width=#3]{Figures/#1}}\fbox{\includegraphics[width=#3]{Figures/#2}}}
\end{center}
\caption{#4}\label{fig:#1} 
\end{figure}
}

\def\Figuretwo[#1,#2]#3\par{
	\Figuretwow{#1}{#2}{0.48 \hsize}
		#3\par	
}

\def\Figuretwoframed[#1,#2]#3\par{
	\Figuretwowframed{#1}{#2}{0.48 \hsize}
		#3\par	
}

\def\Figurethreew#1#2#3#4#5\par{
\begin{figure}[tb]
\begin{center}{
\includegraphics[width=#4]{Figures/#1} \includegraphics[width=#4]{Figures/#2} \includegraphics[width=#4]{Figures/#3}}
\end{center}
\caption{#5}\label{fig:#1} %\prooflabel{#1}
\end{figure}
}

\def\Figurethree#1#2#3#4\par{
	\Figurethreew{#1}{#2}{#3}{0.3 \hsize}
		{#4}\par	
}

\def\Figurematrixfour#1#2#3#4#5\par{
\begin{figure}[tb]
\begin{center}{
\vbox{\hbox{\includegraphics[width= 0.48 \hsize]{Figures/#1} \includegraphics[width= 0.48 \hsize]{Figures/#2}}\hbox{\includegraphics[width= 0.48 \hsize]{Figures/#3}\includegraphics[width= 0.48 \hsize]{Figures/#4}}}}
\end{center}
\caption{#5}\label{fig:#1} %\prooflabel{#1}
\end{figure}
}


\def\questionequals{= \!\!\!\!\!\!{\scriptstyle ? \atop }\,\,\,}

\def\elll#1{\ell^{\lambda,#1}}
\def\elllp{\ell^{\lambda,p}}
\def\elllRp{\ell^{(\lambda,R),p}}


\def\elllRpz_#1{\ell_{#1{\rm z}}^{(\lambda,R),p}}


\def\sopmatrix#1{\begin{pmatrix}#1\end{pmatrix}}

\def\Proof{\begin{proof}}
\def\mqed{\end{proof}}

\gdef\reffilename{\jobname}
\def\References{\bibliography{\reffilename}}

\outer\def\ends{ 
\end{document}
}


\begin{document}



\textbf{Numerical Analysis MATH50003 (2024\ensuremath{\endash}25) Problem Sheet 5}

\textbf{Problem 1} Compute the LU factorisation (if possible) and the PLU factorisation, where the entry of largest magnitude is always permuted to the diagonal, of the following matrices:
\[
\begin{bmatrix}
1 & 2 & 0 \\
3 & 1 & 2 \\
0 & 5 & 1
\end{bmatrix}, \begin{bmatrix}
0 &  5 & 5 & 5 \\
1 & 2 & 0 & 0 \\
3 & 3 & 3 & 0 \\
0 & 0  & 3 & 1
\end{bmatrix}
\]
\textbf{SOLUTION}

\emph{Matrix 1} For the LU factorisation we have:
\[
A = \begin{bmatrix}
1 & 2 & 0 \\
3 & 1 & 2 \\
0 & 5 & 1
\end{bmatrix}= \underbrace{\begin{bmatrix}
1 \\
3 & 1 \\
0 & 0 & 1
\end{bmatrix}}_{L_1}   \begin{bmatrix}
1 & 2 & 0 \\
 & -5 & 2 \\
 & 5 & 1
\end{bmatrix}
\]
We now have
\[
A_2 = \begin{bmatrix}
  -5 & 2 \\
  5 & 1 \end{bmatrix} = 
  \underbrace{\begin{bmatrix}
    1\\
    -1 & 1
\end{bmatrix}}_{L_2}   \begin{bmatrix}
  -5 & 2 \\
   & 3 \end{bmatrix}
\]
We can put it together to find:
\[
A = \underbrace{\begin{bmatrix}
1 \\
3 & 1 \\
0 & -1 & 1
\end{bmatrix}}_{L} \underbrace{\begin{bmatrix}
1 & 2 & 0\\
 & -5 & 2 \\
0 & 0 & 3
\end{bmatrix}}_{U}
\]
For the PLU factorisation we need to permute the largest entry to the diagonal each stage:
\[
\underbrace{\begin{bmatrix} 0 & 1 \\
1 & 0 \\
&& 1 \end{bmatrix}}_{P_1} A = \begin{bmatrix}
3 & 1 & 2 \\
1 & 2 & 0 \\
0 & 5 & 1
\end{bmatrix} = \underbrace{\begin{bmatrix}
1 &  &  \\
1/3 & 1 &  \\
0 & 0 & 1
\end{bmatrix}}_{L_1} \begin{bmatrix}
3 & 1 & 2 \\
1 & 5/3 & -2/3 \\
0 & 5 & 1
\end{bmatrix}
\]
We now permute again since $5 > 5/3$:
\[
\underbrace{\begin{bmatrix} 0 & 1 \\
1 & 0 \end{bmatrix}}_{P_2} A_2 = \begin{bmatrix}5 & 1 \\
5/3 & -2/3 \end{bmatrix} = \underbrace{\begin{bmatrix}1 \\
1/3 & 1 \end{bmatrix}}_{L_2} \begin{bmatrix} 5 & 1 \\ & -1 \end{bmatrix}
\]
We thus have:
\[
P_1 A = L_1 \begin{bmatrix} 1 \\ & P_2^\ensuremath{\top} L_2 \end{bmatrix} \underbrace{\begin{bmatrix}  3 & 1 & 2 \\ 
    & 5 & 1 \\ && -1 \end{bmatrix}}_U = 
    = \begin{bmatrix} 1 \\ & P_2^\ensuremath{\top} \end{bmatrix} 
    \underbrace{\begin{bmatrix} 1 \\ 
    0 & 1 \\
    1/3 & 1/3 & 1 \end{bmatrix}}_L U,
\]
that is,
\[
P = \begin{bmatrix} 1 \\ & P_2 \end{bmatrix} P_1 = \begin{bmatrix}
0 & 1 & 0 \\
0 & 0 & 1 \\
1 & 0 & 0
\end{bmatrix}
\]
\emph{Matrix 2}

This has no LU factorisation since the first entry is 0 so we only deduce the PLU. First permute the largest entry to the diagonal by a simple swap and factor:
\[
\underbrace{\begin{bmatrix}
0 &  & 1 \\
 & 1 \\
 1 & & 0 \\
 &&& 1
\end{bmatrix}}_{P_1} A = \begin{bmatrix} 3 & 3 &3 \\
1 & 2 \\
 & 5  & 5 & 5 \\
 && 3 & 1 
 \end{bmatrix} = 
  \underbrace{\begin{bmatrix} 1 &  & \\
1/3 & 1 \\
 &   &  1 \\
 &&  & 1 
 \end{bmatrix}}_{L_1}
  \begin{bmatrix} 3 & 3 &3 \\
 & 1 & -1 \\
 & 5  & 1 & 2 \\
 && 3 & 1 
 \end{bmatrix}
\]
We repeat with the subslice:
\[
\underbrace{\begin{bmatrix}
0 & 1 &  \\
1 & 0 \\
  & & 1 \\
\end{bmatrix}}_{P_2}  A_2 =
  \begin{bmatrix} 
  5  & 5 & 5 \\
  1 & -1 \\
 & 3 & 1 
 \end{bmatrix} = 
   \underbrace{\begin{bmatrix} 
  1  &  &  \\
  1/5 & 1 \\
0 &  & 1 
 \end{bmatrix}}_{L_2}   \begin{bmatrix} 
  5  & 5 & 5 \\
   & -2 & -1 \\
 & 3 & 1 
 \end{bmatrix}
\]
Finally, we have
\[
\underbrace{\begin{bmatrix}
0 & 1   \\
1 & 0 
\end{bmatrix}}_{P_3}  A_3 = \begin{bmatrix}3 & 1 \\ -2 & -1 \end{bmatrix}
= \underbrace{\begin{bmatrix}1 &  \\ -2/3 & 1 \end{bmatrix}}_{L_3}
\underbrace{\begin{bmatrix}3 & 1 \\  & -1/3 \end{bmatrix}}_{U_3}
\]
We now need to swap the lower matrices and permutations which we do one step at a time.  We already know $A_3 = P_3^\ensuremath{\top} L_3 U_3$ which tells us that
\meeq{
A_2 = P_2^\ensuremath{\top} L_2 \begin{bmatrix} \ensuremath{\alpha}_2 & \ensuremath{\bm{\w}}_2^\ensuremath{\top} \\ & A_3 \end{bmatrix} = 
P_2^\ensuremath{\top} \begin{bmatrix} 1 \\ & P_3^\ensuremath{\top} \end{bmatrix} \begin{bmatrix}
1 \\
P_3 \ensuremath{\bm{\v}}_2/\ensuremath{\alpha}_2 & L_3 \end{bmatrix}  \begin{bmatrix} \ensuremath{\alpha}_2 & \ensuremath{\bm{\w}}_2^\ensuremath{\top} \\ & U_3 \end{bmatrix} \ccr
= \underbrace{\begin{bmatrix}
0 & 0 & 1 \\
1 & 0 & 0 \\
0 & 1 & 0
\end{bmatrix}}_{\tilde{P}_2^\ensuremath{\top}} \underbrace{\begin{bmatrix} 1 \\ 
0 & 1 \\
1/5 & -2/3 & 1 \end{bmatrix}}_{\tilde{L}_2} \underbrace{\begin{bmatrix} 
  5  & 5 & 5 \\
   & 3 & 1 \\
 &  & -1/3 
 \end{bmatrix}}_{\tilde{U}_2}
}
Finally,
\meeq{
A = P_1^\ensuremath{\top} L_1 \begin{bmatrix} \ensuremath{\alpha}_1 & \ensuremath{\bm{\w}}_1^\ensuremath{\top} \\ & A_2 \end{bmatrix} = 
P_1^\ensuremath{\top} \begin{bmatrix} 1 \\ & \tilde{P}_2^\ensuremath{\top} \end{bmatrix} \begin{bmatrix}
1 \\
\tilde{P}_2 \ensuremath{\bm{\v}}_1/\ensuremath{\alpha}_1 & \tilde{L}_2 \end{bmatrix}  \begin{bmatrix} \ensuremath{\alpha}_1 & \ensuremath{\bm{\w}}_1^\ensuremath{\top} \\ & \tilde{U}_2 \end{bmatrix} \ccr
= \underbrace{\begin{bmatrix}
0 & 1 & 0 & 0 \\
0 & 0 & 0 & 1\\
1 & 0 & 0 & 0 \\
0 & 0 & 1 & 0
\end{bmatrix}}_{P^\ensuremath{\top}}   \underbrace{\begin{bmatrix}1 \\
1/3 & 1 \\ 
0 &  0 & 1 \\
1& 1/5 & -2/3 & 1 \end{bmatrix}}_{L} \underbrace{\begin{bmatrix} 
 3 & 3 &3 & 0\\
  &5  & 5 & 5 \\
   && 3 & 1 \\
 &  && -2/3 
 \end{bmatrix}}_{U}
}
\textbf{END}

\rule{\textwidth}{1pt}
\textbf{Problem 2} By computing the Cholesky factorisation, determine which of the following matrices are symmetric positive definite:
\[
\begin{bmatrix} 1 & -1  \\
-1 & 3
\end{bmatrix}, \begin{bmatrix} 1 & 2 & 2  \\
2 & 1 & 2\\
2 & 2 & 1
\end{bmatrix},
\begin{bmatrix} 4 & 2 & 2 & 1  \\
2 & 4 & 2 & 2\\
2 & 2 & 4 & 2 \\
1 & 2 & 2 & 4
\end{bmatrix}
\]
\textbf{SOLUTION}

A matrix is symmetric positive definite (SPD) if and only if it has a Cholesky factorisation, so the task here is really just to compute Cholesky factorisations (by hand). Since our goal is to tell if the Cholesky factorisations exist, we do not have to compute $L_k$'s. We only need to see if the factorisation process can continue to the end.

\emph{Matrix 1}
\[
A_0=\begin{bmatrix} 1 & -1  \\
-1 & 3
\end{bmatrix}
\]
and     $A_1=3-\frac{(-1)\ensuremath{\times}(-1)}{1}>0$, so Matrix 1 is SPD.

\emph{Matrix 2}
\[
A_0=\begin{bmatrix}
1 & 2 & 2 \\
2 & 1 & 2 \\
2 & 2 & 1
\end{bmatrix}
\]
Then
\[
A_1=\begin{bmatrix}
1&2\\
2&1
\end{bmatrix}-\begin{bmatrix} 2 \\ 2 \end{bmatrix}\begin{bmatrix} 2 & 2 \end{bmatrix}=
\begin{bmatrix}
-3&-2\\
-2&-3
\end{bmatrix}
\]
and finally $A_1[1,1] \ensuremath{\leq} 0$, so Matrix 2 is not SPD.

\emph{Matrix 3}
\[
A_0=\begin{bmatrix}
4 & 2 & 2 & 1 \\
2 & 4 & 2 & 2 \\
2 & 2 & 4 & 2 \\
1 & 2 & 2 & 4
\end{bmatrix}
\]
and then
\[
A_1=\begin{bmatrix}
4&2&2\\
2&4&2\\
2&2&4
\end{bmatrix}-\frac{1}{4}\begin{bmatrix} 2 \\ 2 \\ 1 \end{bmatrix}\begin{bmatrix} 2 & 2 & 1 \end{bmatrix}=\frac{1}{4}
\begin{bmatrix}
12&4&6\\
4&12&6\\
6&6&15
\end{bmatrix}
\]
Furthermore
\[
4A_2=\begin{bmatrix}
12&6\\
6&15
\end{bmatrix}-\frac{1}{12}\begin{bmatrix} 4 \\ 6 \end{bmatrix}\begin{bmatrix} 4 & 6 \end{bmatrix}=\frac{4}{3}
\begin{bmatrix}
8&3\\
3&9
\end{bmatrix}
\]
and finally $3A_3=9-\frac{3\ensuremath{\times} 3}{8}>0$, so Matrix 4 is SPD.

\textbf{END}

\textbf{Problem 3} Show that a matrix $A \ensuremath{\in} \ensuremath{\bbR}^{n \ensuremath{\times} n}$ is symmetric positive definite if and only if it has a \emph{reverse} Cholesky factorisation of the form
\[
A = U U^\ensuremath{\top}
\]
where $U$ is upper triangular with positive entries on the diagonal.

\textbf{SOLUTION}

Note $\ensuremath{\bm{\x}}^\ensuremath{\top} U U^\ensuremath{\top} \ensuremath{\bm{\x}} = \| U^\ensuremath{\top} \ensuremath{\bm{\x}} \| > 0$ since $U$ is invertible.

For the other direction, we replicate the proof by induction for standard Cholesky, beginning in the bottom right instead of the top left. Again the basis case is trivial. Since all diagonal entries are positive we can write
\[
A = \begin{bmatrix} K & \ensuremath{\bm{\v}}\\
                    \ensuremath{\bm{\v}}^\ensuremath{\top} & \ensuremath{\alpha} \end{bmatrix} =
                    \underbrace{\begin{bmatrix} I & {\ensuremath{\bm{\v}} \over \sqrt{\ensuremath{\alpha}}} \\
                                        & \sqrt{\ensuremath{\alpha}}
                                        \end{bmatrix}}_{U_1}
                    \begin{bmatrix} K - {\ensuremath{\bm{\v}} \ensuremath{\bm{\v}}^\ensuremath{\top} \over \ensuremath{\alpha}}  & \\
                     & 1 \end{bmatrix}
                     \underbrace{\begin{bmatrix} I \\
                      {\ensuremath{\bm{\v}}^\ensuremath{\top} \over \sqrt{\ensuremath{\alpha}}} & \sqrt{\ensuremath{\alpha}}
                                        \end{bmatrix}}_{U_1^\ensuremath{\top}}
\]
By assumption $K - {\ensuremath{\bm{\v}} \ensuremath{\bm{\v}}^\ensuremath{\top} \over \ensuremath{\alpha}} = \Ut\Ut^\ensuremath{\top}$ hence we have
\[
A = \underbrace{U_1 \begin{bmatrix} \Ut \\ & 1 \end{bmatrix}}_U  \underbrace{\begin{bmatrix} \Ut^\top \\ & 1 \end{bmatrix} U_1^\top}_{U^\top}
\]
\textbf{END}

\textbf{Problem 4(a)} Use the Cholesky factorisation to prove that the following $n \ensuremath{\times} n$ matrix is symmetric positive definite for any $n$:
\[
\ensuremath{\Delta}_n := \begin{bmatrix}
2 & -1 \\
-1 & 2 & -1 \\
& -1 & 2 & \ensuremath{\ddots} \\
&& \ensuremath{\ddots} & \ensuremath{\ddots} & -1 \\
&&& -1 & 2
\end{bmatrix}
\]
Hint: consider a matrix $K_n^{(\ensuremath{\alpha})}$ that equals $\ensuremath{\Delta}_n$ apart from the top left entry which is $\ensuremath{\alpha} > 1$ and use a proof by induction.

\textbf{SOLUTION}

Consider the first step of the Cholesky factorisation:
\[
\ensuremath{\Delta}_n = \begin{bmatrix} 2 & -\ensuremath{\bm{\e}}_1^\ensuremath{\top} \\
                    -\ensuremath{\bm{\e}}_1 & \ensuremath{\Delta}_{n-1} \end{bmatrix} =
                    \underbrace{\begin{bmatrix} \sqrt{2} \\
                                    {-\ensuremath{\bm{\e}}_1 \over \sqrt{2}} & I
                                        \end{bmatrix}}_{L_1}
                    \begin{bmatrix}1 \\ & \ensuremath{\Delta}_{n-1} - {\ensuremath{\bm{\e}}_1 \ensuremath{\bm{\e}}_1^\ensuremath{\top} \over 2} \end{bmatrix}
                    \underbrace{\begin{bmatrix} \sqrt{2} & {-\ensuremath{\bm{\e}}_1^\ensuremath{\top} \over \sqrt{2}} \\
                                                            & I
                                        \end{bmatrix}}_{L_1^\ensuremath{\top}}
\]
The bottom right is merely $\ensuremath{\Delta}_{n-1}$ but with a different $(1,1)$ entry! This hints at a strategy of proving by induction.

Assuming $\ensuremath{\alpha} > 1$ write
\[
K_n^{(\ensuremath{\alpha})} := \begin{bmatrix}
\ensuremath{\alpha} & -1 \\
-1 & 2 & -1 \\
& -1 & 2 & \ensuremath{\ddots} \\
&& \ensuremath{\ddots} & \ensuremath{\ddots} & -1 \\
&&& -1 & 2
\end{bmatrix} =
                    \begin{bmatrix} \sqrt{\ensuremath{\alpha}} \\
                                    {-\ensuremath{\bm{\e}}_1 \over \sqrt{\ensuremath{\alpha}}} & I
                                        \end{bmatrix}
                    \begin{bmatrix}1 \\ & K_{n-1}^{(2 - 1/\ensuremath{\alpha})} \end{bmatrix}
                    \begin{bmatrix} \sqrt{\ensuremath{\alpha}} & {-\ensuremath{\bm{\e}}_1^\ensuremath{\top} \over \sqrt{\ensuremath{\alpha}}} \\
                                                            & I
                                        \end{bmatrix}
\]
Note if $n = 1$ this is trivially SPD. Hence assume $K_{n-1}^{(\ensuremath{\alpha})}$ is SPD for all $\ensuremath{\alpha} > 1$. If $\ensuremath{\alpha} > 1$ then $2 - 1/\ensuremath{\alpha} > 1$. Hence by induction and the fact that $\ensuremath{\Delta}_n = K_n^{(2)}$ we conclude that $\ensuremath{\Delta}_n$ has a Cholesky factorisation and hence is symmetric positive definite.

\textbf{END}

\textbf{Problem 4(b)} Deduce its Cholesky factorisations: $\ensuremath{\Delta}_n = L_n L_n^\ensuremath{\top}$ where $L_n$ is lower triangular.

\textbf{SOLUTION}

We can  write down the factors explicitly: define $\ensuremath{\alpha}_1 := 2$ and
\[
\ensuremath{\alpha}_{k+1} = 2- 1/\ensuremath{\alpha}_k.
\]
Let's try out the first few:
\[
\ensuremath{\alpha}_1 = 2, \ensuremath{\alpha}_2 = 3/2, \ensuremath{\alpha}_3 = 4/3, \ensuremath{\alpha}_4 = 5/4, \ensuremath{\ldots}
\]
The pattern is clear and one can show by induction that $\ensuremath{\alpha}_k = (k+1)/k$. Thus we have the Cholesky factorisation
\meeq{
\ensuremath{\Delta} _n = \underbrace{\begin{bmatrix}
\sqrt{2} \\
-1/\sqrt{2} & \sqrt{3/2} \\
& -\sqrt{2/3} & \sqrt{4/3} \\
    && \ensuremath{\ddots} & \ensuremath{\ddots} \\
    &&& -\sqrt{(n-1)/n} & \sqrt{(n+1)/n}
    \end{bmatrix}}_{L_n}  \\
    & \qquad \ensuremath{\times}     \underbrace{\begin{bmatrix}
\sqrt{2} & -1/\sqrt{2} \\
 & \sqrt{3/2} & -\sqrt{2/3} \\
    && \ensuremath{\ddots} & \ensuremath{\ddots} \\
    &&& \sqrt{n/(n-1)} & -\sqrt{(n-1)/n} \\
    &&&& \sqrt{(n+1)/n}
    \end{bmatrix}}_{L_n^\ensuremath{\top}}
}
\textbf{END}



\end{document}