
\section{Cholesky factorisation}
In the special case where $A$ is  a real square \emph{symmetric positive definite} (SPD, that is $A \ensuremath{\in} \ensuremath{\bbR}^{n \ensuremath{\times} n}$ such that $A^\ensuremath{\top} = A$ and $\ensuremath{\bm{\x}}^\ensuremath{\top} A \ensuremath{\bm{\x}} > 0$ for all $\ensuremath{\bm{\x}} \ensuremath{\in} \ensuremath{\bbR}^n$, $\ensuremath{\bm{\x}} \ensuremath{\neq} 0$) matrix the LU factorisation has a special form called the \emph{Cholesky factorisation}:
\[
A = L L^\ensuremath{\top},
\]
i.e., $U = L^\ensuremath{\top}$, but now $L$ does not necessarily have 1 on the diagonal. This provides an algorithmic way to \emph{prove} that a matrix is symmetric positive definite, and is roughly twice as fast as the LU factorisation to compute.

\begin{definition}[positive definite] A square matrix $A \ensuremath{\in} \ensuremath{\bbR}^{n \ensuremath{\times} n}$ is \emph{positive definite} if for all $\ensuremath{\bm{\x}} \ensuremath{\in} \ensuremath{\bbR}^n, x \ensuremath{\neq} 0$ we have
\[
\ensuremath{\bm{\x}}^\ensuremath{\top} A \ensuremath{\bm{\x}} > 0
\]
\end{definition}

First we establish some basic properties of positive definite matrices:

\begin{proposition}[conjugating positive definite] If  $A \ensuremath{\in} \ensuremath{\bbR}^{n \ensuremath{\times} n}$ is positive definite and $V \ensuremath{\in} \ensuremath{\bbR}^{n \ensuremath{\times} n}$ is non-singular then
\[
V^\ensuremath{\top} A V
\]
is positive definite. \end{proposition}
\textbf{Proof}

For all  $\ensuremath{\bm{\x}} \ensuremath{\in} \ensuremath{\bbR}^n, \ensuremath{\bm{\x}} \ensuremath{\neq} 0$, define $\ensuremath{\bm{\y}} = V \ensuremath{\bm{\x}} \ensuremath{\neq} 0$ (since $V$ is non-singular). Thus we have
\[
\ensuremath{\bm{\x}}^\ensuremath{\top} V^\ensuremath{\top} A V \ensuremath{\bm{\x}} = \ensuremath{\bm{\y}}^\ensuremath{\top} A \ensuremath{\bm{\y}} > 0.
\]
\ensuremath{\QED}

\begin{proposition}[diag positivity] If $A \ensuremath{\in} \ensuremath{\bbR}^{n \ensuremath{\times} n}$ is positive definite then its diagonal entries are positive: $a_{kk} > 0$. \end{proposition}
\textbf{Proof}
\[
a_{kk} = \ensuremath{\bm{\e}}_k^\ensuremath{\top} A \ensuremath{\bm{\e}}_k > 0.
\]
\ensuremath{\QED}

\begin{lemma}[subslice positive definite] If $A \ensuremath{\in} \ensuremath{\bbR}^{n \ensuremath{\times} n}$ is positive definite then $A[2:n,2:n] \ensuremath{\in} \ensuremath{\bbR}^{(n-1) \ensuremath{\times} (n-1)}$ is also positive definite. \end{lemma}
\textbf{Proof} For all $\ensuremath{\bm{\x}} \ensuremath{\in} \ensuremath{\bbR}^{n-1}$, define $\ensuremath{\bm{\y}} := [0,\ensuremath{\bm{\x}}]$. Then we have
\[
\ensuremath{\bm{\x}}^\ensuremath{\top} A[2:n,2:n] \ensuremath{\bm{\x}} = \ensuremath{\bm{\y}}^\ensuremath{\top} A \ensuremath{\bm{\y}} > 0.
\]
\ensuremath{\QED}

Here is the key result:

\begin{theorem}[Cholesky and SPD] A matrix $A$ is symmetric positive definite if and only if it has a Cholesky factorisation
\[
A = L L^\ensuremath{\top}
\]
where $L$ is lower triangular with positive diagonal entries.

\end{theorem}
\textbf{Proof} If $A$ has a Cholesky factorisation it is symmetric ($A^\ensuremath{\top} = (L L^\ensuremath{\top})^\ensuremath{\top} = A$) and for $\ensuremath{\bm{\x}} \ensuremath{\neq} 0$ we have
\[
\ensuremath{\bm{\x}}^\ensuremath{\top} A \ensuremath{\bm{\x}} = (L^\ensuremath{\top}\ensuremath{\bm{\x}})^\ensuremath{\top} L^\ensuremath{\top} \ensuremath{\bm{\x}} = \|L^\ensuremath{\top}\ensuremath{\bm{\x}}\|^2 > 0
\]
where we use the fact that $L$ is non-singular.

For the other direction we will prove it by induction, with the $1 \ensuremath{\times} 1$ case being trivial. Assume all lower dimensional symmetric positive definite matrices have Cholesky decompositions. Modifying the LU factorisation slightly we write
\[
A = \begin{bmatrix} \ensuremath{\alpha} & \ensuremath{\bm{\v}}^\ensuremath{\top} \\
                    \ensuremath{\bm{\v}}   & K
                    \end{bmatrix} = \underbrace{\begin{bmatrix} \sqrt{\ensuremath{\alpha}} \\
                                    {\ensuremath{\bm{\v}} \over \sqrt{\ensuremath{\alpha}}} & I \end{bmatrix}}_{L_1}
                                    \begin{bmatrix} 1  \\ & K - {\ensuremath{\bm{\v}} \ensuremath{\bm{\v}}^\ensuremath{\top} \over \ensuremath{\alpha}} \end{bmatrix}
                                    \underbrace{\begin{bmatrix} \sqrt{\ensuremath{\alpha}} & {\ensuremath{\bm{\v}}^\ensuremath{\top} \over \sqrt{\ensuremath{\alpha}}} \\
                                     & I \end{bmatrix}}_{L_1^\ensuremath{\top}}.
\]
Note that $A_2 := K - {\ensuremath{\bm{\v}} \ensuremath{\bm{\v}}^\ensuremath{\top} \over \ensuremath{\alpha}}$ is a subslice of $L_1^{-1} A L_1^{-\ensuremath{\top}}$, hence by combining the previous propositions is itself SPD. Thus we can write
\[
A_2 = K - {\ensuremath{\bm{\v}} \ensuremath{\bm{\v}}^\ensuremath{\top} \over \ensuremath{\alpha}} = L_2 L_2^\ensuremath{\top}
\]
and hence $A = L L^\ensuremath{\top}$ for
\[
L= L_1 \begin{bmatrix}1 \\ & L_2 \end{bmatrix} = \begin{bmatrix} \sqrt{\ensuremath{\alpha}} \\ {\ensuremath{\bm{\v}} \over \sqrt{\ensuremath{\alpha}}} & L_2 \end{bmatrix}.
\]
\ensuremath{\QED}

\begin{example}[Cholesky by hand] Consider the matrix
\[
A = \begin{bmatrix}
2 &1 &1 &1 \\
1 & 2 & 1 & 1 \\
1 & 1 & 2 & 1 \\
1 & 1 & 1 & 2
\end{bmatrix}
\]
Then $\ensuremath{\alpha}_1 = 2$, $\ensuremath{\bm{\v}}_1 = [1,1,1]$, and
\[
A_2 = \begin{bmatrix}
2 &1 &1 \\
1 & 2 & 1 \\
1 & 1 & 2
\end{bmatrix} - {1 \over 2} \begin{bmatrix} 1 \\ 1 \\ 1 \end{bmatrix} \begin{bmatrix} 1 & 1 & 1 \end{bmatrix}
={1 \over 2} \begin{bmatrix}
3 & 1 & 1 \\
1 & 3 & 1 \\
1 & 1 & 3
\end{bmatrix}.
\]
Continuing, we have $\ensuremath{\alpha}_2 = 3/2$, $\ensuremath{\bm{\v}}_2 = [1/2,1/2]$, and
\[
A_3 = {1 \over 2} \left( \begin{bmatrix}
3 & 1 \\ 1 & 3
\end{bmatrix} - {1 \over 3} \begin{bmatrix} 1 \\ 1  \end{bmatrix} \begin{bmatrix} 1 & 1  \end{bmatrix}
\right)
= {1 \over 3} \begin{bmatrix} 4 & 1 \\ 1 & 4 \end{bmatrix}
\]
Next, $\ensuremath{\alpha}_3 = 4/3$, $\ensuremath{\bm{\v}}_3 = [1]$, and
\[
A_4 = [4/3 - 3/4 * (1/3)^2] = [5/4]
\]
i.e. $\ensuremath{\alpha}_4 = 5/4$.

Thus we get
\[
L= \begin{bmatrix}
\sqrt{\ensuremath{\alpha}_1} \\
{\ensuremath{\bm{\v}}_1[1] \over \sqrt{\ensuremath{\alpha}_1}} & \sqrt{\ensuremath{\alpha}_2} \\
{\ensuremath{\bm{\v}}_1[2] \over \sqrt{\ensuremath{\alpha}_1}} & {\ensuremath{\bm{\v}}_2[1] \over \sqrt{\ensuremath{\alpha}_2}}  & \sqrt{\ensuremath{\alpha}_3} \\
{\ensuremath{\bm{\v}}_1[3] \over \sqrt{\ensuremath{\alpha}_1}} & {\ensuremath{\bm{\v}}_2[2] \over \sqrt{\ensuremath{\alpha}_2}}  & {\ensuremath{\bm{\v}}_3[1] \over \sqrt{\ensuremath{\alpha}_3}}  & \sqrt{\ensuremath{\alpha}_4}
\end{bmatrix}
 = \begin{bmatrix} \sqrt{2} \\ {1 \over \sqrt{2}} & \sqrt{3 \over 2} \\
{1 \over \sqrt{2}} & {1 \over \sqrt 6} & {2 \over \sqrt{3}} \\
{1 \over \sqrt{2}} & {1 \over \sqrt 6} & {1 \over \sqrt{12}} & {\sqrt{5} \over 2}
\end{bmatrix}
\]
\end{example}



